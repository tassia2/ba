\section{Numerical Simulation in Urban Environments}
As an illustration of the methodology of applying mathematical modeling,
numerical simulation and scientific visualization for complex simulations
made possible by \hiflow{}, we present the Karlsruhe Geometry project
which performs numerical simulations in an urban environment in
collaboration with the city council of Karlsruhe, Germany.  Using the increasing
detail of three dimensional city models, more reliable simulations are
feasible, for example concerning airflow, fine-dust or noise distribution
in the very environment that we live in.

This section describes an example of airflow around the
Department of Mathematics of the Karlsruhe Institute of
Technology, in order to illustrate how the components
of \hiflow{} are combined into a simulation.

\begin{figure}[ht]
  \centering
  \includegraphics[width=0.45\textwidth,height=0.253\textwidth]{fig/simulationen-im-karlsruher-stadtmodell-2-schrift}
  %\hspace{1em}
  \includegraphics[width=0.45\textwidth,height=0.253\textwidth]{fig/simulationen-im-karlsruher-stadtmodell-klein-schrift}
  \caption{Geometry used for the simulation and a visualization of simulation results.}
  \label{fig:KarlsruheGeometry}
\end{figure}

\subsection{Problem Formulation}
The instationary Navier-Stokes equations are solved in a box
surrounding the buildings of the Kronenplatz square. As boundary
conditions, a simple Poiseuille profile is used on one side
$\Gamma_{in}$, and natural do-nothing boundary conditions are used on
the opposite side $\Gamma_{out}$. On the remaining boundaries,
including the walls of the buildings, the velocity is set to zero. The
model is formulated as an initial boundary value problem for the
velocity $\vec{u}\left(\vec{x}, t\right)$ and the pressure
$p\left(\vec{x}, t\right)$ in Equation \ref{eq:NS_stark_instat}.

\begin{equation}
  \begin{aligned}
    \partial_t{\vec{u}} - \nu \Delta \vec{u} + \left(\vec{u} \cdot \nabla \right) \vec{u} + \nabla p & = 0 & \left(\vec{x}, t\right) \in \Omega \times \left(0,T\right)\ ,\\
    \nabla \cdot \vec{u} & = 0 & \left(\vec{x}, t\right) \in \Omega \times \left(0,T\right)\ ,\\ 
    \vec{u} & = \vec{u}^{in} & \left(\vec{x}, t\right) \in \Gamma_{\text{in}} \times \left(0,T\right)\ ,\\ 
    \left(-\mathcal{I} p + \nu \nabla \vec{u} \right) \cdot \vec{n} & = 0 & \left(\vec{x}, t\right) \in \Gamma_{\text{out}} \times \left(0,T\right)\ ,\\ 
    \vec{u} & = 0  & \left(\vec{x}, t\right) \in \Gamma \times \left(0,T\right)\ ,\\
    \vec{u}\left(\vec{x}, 0\right) & = \vec{u}_0\left(\vec{x}\right) &\vec{x} \in \Omega\ .
    \label{eq:NS_stark_instat}      
  \end{aligned}  
\end{equation}   

\subsection{Implementation of the Navier-Stokes Solver}

Equation \ref{eq:NS_stark_instat} is discretized first in time, and
then in space. For the time discretization, the fractional step
$\theta$-scheme is used, and in space $Q_2-Q_1$ (``Taylor-Hood'')
finite elements~\cite{brennerscott2002}. % TODO check reference for fractional-step method: TurekRHG06
Due to the nonlinearity
$\left(\vec{u} \cdot \nabla \right) \vec{u}$ each timestep is resolved
with the exact Newton algorithm. This involves solving a linearized
problem iteratively, which is done using the GMRES method
preconditioned by the ILU++ \cite{iluplusplus} preconditioner. Overall,
the computation is structured as in Algorithm~\ref{algo:inst_ns}, with
the supporting modules indicated in parentheses.

\begin{algorithm}
  \caption{Overall structure of the instationary Navier-Stokes simulation}
  \label{algo:inst_ns}
  \begin{algorithmic}
    \Function{solve\_instationary\_navier\_stokes}{mesh\_filename}
    \State Read mesh and distribute to all processors (Mesh).
    \State Number the DoFs using a Q2-Q1 finite element space (DoF/FEM).
    \State Create the system matrix, residual vector and solution vector (Linear Algebra).
    \State Initialize the solution vector with initial solution 0.
    \State Set the DoF values on the inflow boundary to the values prescribed by $\vec{u}^{in}$.
    \While{$t < T$}
      \For{$s = 1,2,3$}
        \State Solve nonlinear problem for substep $s$ to obtain $\left(\vec{u}^{n,s}, p^{n,s}\right)$.
      \EndFor
    \State $t = t + \Delta{}t$
    \State Visualize the solution at timestep $t$
    \EndWhile
    \EndFunction
  \end{algorithmic}
\end{algorithm}
%
The nonlinear solution algorithm is encapsulated in the class
\code{Newton} which implements an abstract interface
\code{NonlinearSolver}. This has a reference to an object which
implements the functions \code{EvalGrad} and \code{EvalFunc} that
compute the system matrix and the residual vector, respectively. These
two functions use the assembly functionality described in
Section~\ref{sec:assembly}. The \code{NonlinearSolver} contains a
reference to a linear solver of type \code{GMRES}, which is used in each step to compute
the update for the current linearization point.

The local assembly of the linearized variational formulation is
implemented in a class \code{InstationaryFlowAssembler}. The state of
this class depends on the current substep in the timestepping method,
as well as the solutions corresponding to the linearization point in
the Newton method and the previous timestep. This problem-specific
code is relatively long, but easy to write once the corresponding
variational form has been derived analytically.

This example illustrates the interplay between reusable software
components, provided in the \hiflow{} library through the three core 
modules, and the problem-specific code that must be implemented
specifically for each individual solver. In this case, the Mesh module
provides the functionality for reading in and distributing the
computational mesh; the DoF/FEM module computes the DoF numbering, and
is also used in the assembly of the system matrix and residual; and
the Linear Algebra module provides management of matrices and vectors,
as well as an iterative method to solve the linear system. Abstract
interfaces are used in several places, both for implementing callbacks
for the components in the library, such as for the nonlinear solver;
and for facilitating the exchange of components, as with the
concrete linear and nonlinear solvers (GMRES and Newton).


%%% NOT USED - but maybe we need it some day?
% For the time discretization, a
% fractional step $\theta$-scheme method is used, in which each timestep
% in split into three substeps. The PDE:s solved when advancing from
% timestep $t_n$ to timestep $t_{n+1}$ are detailed in Equations
% \ref{eq:substep1} - \ref{eq:substep3}, where the operator
% $A\left(\vec{u}\right) = -\nu\Delta{} +
% \left(\vec{u}\cdot{}\nabla{}\right)$ has been introduced to shorten the notation.

% \begin{equation}
%   \begin{aligned}
%     \left(\mathcal{I} + \tilde{\theta}A\left(\vec{u}^{n+\theta}\right)\right)\vec{u}^{n+\theta} + \nabla{}p^{n+\theta} 
%     & = & \left(\mathcal{I} - \gamma\theta\Delta{t}A\left(\vec{u}^{n}\right)\right)\vec{u}^{n} \\
%     \nabla \cdot \vec{u^{n+\theta}} & = & 0
%     \end{aligned}
%     \label{eq:substep1}
% \end{equation}
% \begin{equation}
%   \begin{aligned}
%     \left(\mathcal{I} + \tilde{\theta}A\left(\vec{u}^{n+1-\theta}\right)\right)\vec{u}^{n+1-\theta} + \nabla{}p^{n+1-\theta} 
%     & = & \left(\mathcal{I} - \beta\theta\prime\Delta{t}A\left(\vec{u}^{n+\theta}\right)\right)\vec{u}^{n+\theta} \\
%     \nabla \cdot \vec{u^{n+1-\theta}} & = & 0
%     \end{aligned}
%     \label{eq:substep2}
% \end{equation}
% \begin{equation}
%   \begin{aligned}
%     \left(\mathcal{I} + \tilde{\theta}A\left(\vec{u}^{n+1}\right)\right)\vec{u}^{n+1} + \nabla{}p^{n+1} 
%     & = & \left(\mathcal{I} - \gamma\theta\Delta{t}A\left(\vec{u}^{n+1-\theta}\right)\right)\vec{u}^{n+1-\theta} \\
%     \nabla \cdot \vec{u^{n+1}} & = & 0
%     \end{aligned}
%     \label{eq:substep3}
% \end{equation}

% The parameter $\theta$ is chosen as $\theta = 1-\frac{\sqrt{2}}{2}$,
% and the other coefficients introduced are related as follows:
% $\theta\prime = 1-2\theta$, $\beta = \frac{\theta\prime}{1-\theta}$
