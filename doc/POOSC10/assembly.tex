\section{Assembly Tools}
\label{sec:assembly}
The assembly tools in \hiflow{} combine the functionality in the Mesh,
DoF/FEM and Linear Algebra modules, to provide a mechanism that
efficiently produces the global system matrix and right-hand-side
vector. As is common in finite element literature, the global assembly
algorithm is element-oriented, with an outer iteration over the cells
in the mesh that performs a local assembly on each cell
separately. This section first describes the global assembly
algorithm, which is independent of the PDE, and then
explains how the local assembly can be implemented based on the
variational formulation of the PDE.

\subsection{Global Assembly}
The global assembly algorithm is quite simple, consisting of a loop
over all cells, the computation of the element matrix or vector by the
local assembly, and finally an addition of the result into the global
matrix or vector.

In order to enable optimizations in the local
assembly, the order of iteration of the cells is according to their
type and the polynomial degree of the associated element. This
minimizes the number of times the local basis functions and their
gradients must be evaluated on the reference element.

The local assembly is defined through a user-defined class, on which
the global assembly functions are templated. Inside the loop over the
cells, the global assembly functions first call the function
\code{initialize\_for\_element}, passing the element information and
quadrature object as parameters. The local matrix or vector is then
computed with a call to \code{assemble\_local\_matrix} or
\code{assemble\_local\_vector}, respectively.

The addition of the local element matrix or vector into the
corresponding global object uses the DoF numbering established by the
DoF/FEM module. This numbering is also used to compute the graph of
the global matrix, which is needed for the initialization of the
sparsity structure of this object. The function which does this has
the same structure as the global assembly, and is therefore also a
part of the assembly tools.

% Dealing with the finite element method, each value of a shape function on 
% a degree of freedom needs to be computed and stored in a matrix representing
% the system described by the partial differential equations. To assemble this
% matrix, one needs to consider the variational formulation of a problem,
% which then is evaluated in a first step just on a reference cell and after
% this is transformed into the physical domain. Hence, the task of assembly is
% divided into two parts, namely "global assembly" and "local assembly". This
% subsection deals with the first part.

% Given the information the DoF / FEM module provides, a correspondence between
% reference and physical cell can be established. Therefore, within the global
% assembly, one first needs to iterate over the cells of the mesh (for each
% variable) and do the local assembly on each cell. By the correspondence
% between local and global cell, the next step is to store the local matrix
% at the correct places within the global matrix. Basically, one could say the
% task of global assembly is to manage the local assembly. This routine is not
% meant to be modified by the user. Controlling the assembly procedure from
% the users point of view should be done within the local assembly, which
% will be described next.

\subsection{Local Assembly}
The task of the local assembly is to compute the local element
matrices and vectors. The local assembly is where the variational
formulation of the PDE comes into play. In \hiflow{}, a strategy similar
to that of deal.II \cite{BHK07} is followed, where the user of the
library is responsible for choosing a variational formulation, and
implementing the corresponding local assembler functions in a class,
which we will hereafter call \code{LocalAssembler}, although the
name can be freely chosen in the code. This is in contrast to the approach
described in \cite{AlnaesLoggEtAl2009a}, where a separate language,
UFC, is used to describe the variational formulation. The UFC code is
then compiled to a target language, e.g. C++, and can thereafter be
included in the finite element program. The advantage of such an approach
is a simplified syntax for users of the library, and a potential for
interoperability between finite element libraries. The drawback is, as
with any abstraction layer, that the user has less control over the
code, which could lead to performance penalties.

\hiflow{} supports the user in the creation of the local assembler
through a helper class, which provides a large part of the necessary
functionality. The helper class computes and caches the values of
the local shape functions and their gradients on the reference element
as well as transformed to the physical element. It provides the
information related to the current quadrature rule, and the cell
transformation. Furthermore, it aids in the evaluation of existing
finite element functions defined through coefficient vectors, which is
useful for nonlinear and time-dependent problems. Support for assembly
over facets is also being developed, which is needed e.g.  for
inhomogeneous Neumann boundary conditions.  The helper class works
without user interference on meshes with mixed cell types, varying polynomial
degrees and hanging nodes.

In order to illustrate the construction of a local assembler, Listing
\ref{assembly:laplace} shows C++ code for the Poisson equation
\begin{displaymath}
  - \Delta{}u = x + y
\end{displaymath}
with variational formulation 
\begin{displaymath}
  \int_{\Omega}{\nabla{}u\cdot\nabla{}v{}\textrm{d}\vec{x}} = \int_{\Omega}{(x + y)v{}\textrm{d}\vec{x}}
\end{displaymath}

\lstinputlisting[language=C++, label=assembly:laplace, caption=Local assembler for the Poisson problem]{laplace_assembly.cc}

As illustrated in this code, the helper class \code{AssemblyAssistant}
provides most of the functions needed for the assembly. The user
implements the computation of the local matrix and vector in the
functions \code{assemble\_local\_matrix()} and
\code{assemble\_local\_vector()}, respectively. In this case, the
implementation uses the \code{AssemblyAssistant} class to obtain the
values of the local shape functions and their gradients \code{phi()} and
\code{grad\_phi()}, at the quadrature points of the current quadrature 
rule transformed to the physical element. The function
\code{dof\_index()} provides the position in the matrix or index where
the degrees of freedom corresponding to a given variable (in this case
variable $0$) should be inserted. \code{w(q)} gives the weight
associated with quadrature point \code{q} in the quadrature rule, and
\code{detJ(q)} gives the Jacobian of the element transformation at
that point. The local vector assembly illustrates the use of a
user-defined function \code{my\_f()}, which is evaluated at the point
\code{x(q)} on the physical cell that corresponds to quadrature point
\code{q}. More complex settings, with different models coupled to each
other, can be also be handled in this framework.
