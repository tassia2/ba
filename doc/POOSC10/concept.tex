\section{Motivation, Concepts and Structure of \hiflow{}}

\subsection{Fields of Application}
A typical application incorporating the full complexity of physical
and mathematical modeling is the \emph{United Airways} project
\cite{ua:10}.  Within an interdisciplinary framework its objective is
the simulation of the full human respiratory tract including complex
and time-dependent geometries with different length scales,
turbulence, fluid-structure interaction (e.g.~fine hairs and mucus),
and effects due to temperature and moisture variations. Due to the
complex interaction of all effects it is a great challenge to
construct robust numerical methods giving accurate simulation results
within a moderate time. As a further vision, interactive and real time
simulations should give medical advice on personalized data during
examination and surgery. Another example is the project 
\emph{Goal Oriented Adaptivity for Tropical Cyclones} (see \ref{TropicalCyclones}).
Many weather phenomena such as the development and motion of Tropical
Cyclones are influenced by processes on scales that may range from hundreds 
of meters to thousands of kilometers. Due to computational requirements and memory 
limitations it is often impossible to resolve all relevant scales on globally 
refined meshes. 
Application dependent requirements and varying demands with respect to the quantity 
which is of interest lead to specific treatment of modern adaptive numerical 
methods that need to be adapted to the associated problem settings.

\subsection{Flexibility}
The conceptual goal of \hiflow{} is to be a flexible multi-purpose
software package that can be adapted to most user scenarios. To this
end, the core of \hiflow{} is divided into three main modules: Mesh,
DoF/FEM and Linear Algebra; see
Figure~\ref{softwarekernel}. These three core modules -- further extended
by a suite of other modules -- are essential for the solution
procedure based on finite element methods for partial
differential equations. They offer the main functionality for mapping
mathematical problem scenarios into parallel software.

\begin{figure}[ht]
\centering
\includegraphics[scale=0.4]{./fig/Module_Methoden3}
\caption{Structure of the \hiflow{} core divided into modules and
methods.}
\label{softwarekernel}
\end{figure}

Other building blocks for \hiflow{} are routines for numerical
integration of element integrals, assembling of system matrices,
inclusion of boundary conditions, setting up nonlinear and linear
solvers, providing output data for visualization of solutions, error estimators,
and others. These routines are added to the core modules. This modular
structure ensures the flexibility to employ the library to solve a
wide variety of problems. Another aspect of modularity is to enable
extensibility of the \hiflow{} package.  There exist two basic ways to
extend \hiflow{}. Firstly further modules and methods for user-defined
applications such as modules for stochastic finite elements or
chemical reactions can be added to the \hiflow{} core. Secondly
existing modules and methods can be augmented by adding further
functionality. The possibility to extend \hiflow{} by static or
dynamic polymorphism provides a basis for future development. For
instance in the mesh module a new implementation inherited from the
abstract mesh base class can be implemented, the DoF/FEM module can be
extended by further finite elements spaces such as $H(div)$ or
$H(curl$), and in the Linear Algebra module more data structures for
matrices could be added -- all in the sense of dynamic
polymorphism. Available linear algebra data structure can be extended
through the use of templates. The basic principle of \hiflow{} is the aim for a
generic implementation in order to be able to use different modules
and methods for a variety of problems following the approach of a
multi-purpose software. Of course there is a trade-off when dealing
with problems that need particular implementation and
highly-specialized methods for efficient, robust and accurate
simulations.

\subsection{Performance, Parallelism, Emerging Technologies}
Another object of \hiflow{} is the full utilization of all available
resources on any platform -- from large HPC systems to any stand-alone
workstation or coprocessor-accelerated machine. To achieve this goal
all three modules (Mesh, DoF/FEM, Linear Algebra) provide distributed
data structures. The communication between different nodes and
processors is realized by means of an MPI-layer. On the linear algebra
level we use concepts of hardware-aware computing for acceleration of
basic routines like local matrix-vector and vector-vector
operations. Here, the main challenge is to fully utilize the available
computing power of emerging technologies like multi-core CPUs, 
graphics processing units (GPUs),
multi-GPUs, and any other coprocessor-accelerated or heterogeneous
platform. The advantage of the structure provided by the Linear
Algebra module is the potential to build solvers without having
detailed information on the underlying platform. All methods are layed
out with respect to scalability in a generic sense. Of course, best
performance conflicts with flexibility. In that case \hiflow{} favors
performance over flexibility -- which could mean to offer a solver
which is only efficient for a special problem. Another important issue
is the usability and convenience. \hiflow{} provides the user with a
transparent structure enabling him to set up and get a simulation for 
his application running with minimal effort.

\subsection{Hardware-aware Computing}
The huge demand on fast and accurate simulation results for
large-scale problems poses considerable challenges on the
implementation on modern hardware.  Supercomputers and emerging
parallel hardware like GPUs offer
impressive computing power in the range of Teraflop/s for desktop
supercomputing up to Petaflop/s for cutting edge HPC machines. The
major difficulty is the development of efficient parallel code that is
scalable with respect to exponentially increasing core counts and
portable across a wide range of computing platforms. With the advent
of the many-core era new platforms like the GPUs, the Cell BE,
FPGA-based systems like Convey's Hybrid Core Computer or Intel's tiled
architectures (Polaris, Rock Creek) or the Larrabee incarnation Knight
Ferry have emerged \cite{Convey,williams2,Luk,Oboril10,Schmidtobreick10,williams1,williams3}.
These technologies come along with impressive
capabilities but different programming approaches, different
processing models, and different tool chains. Moreover, all numerical
methods need to be compliant with multiple levels of parallelism
within hybrid systems and with hierarchical memory
subsystems. Typically, manual tuning and parameter variation is
necessary for optimal performance on a dedicated system. In this
context, hardware-aware computing is a multi-disciplinary approach to
identify the best combination of applications, physical models,
numerical schemes, parallel algorithms, and platform-specific
implementations that is giving the fastest and most accurate results
on a particular platform \cite{HAC}. Design goals are maximal
flexibility with respect to software capabilities, mathematical
quality, efficient resource utilization, performance, and portability
of concepts. Hardware-aware computing not only comprises
highly-optimized platform-specific implementations, design of
communication-optimized data structures, and maximizing data reuse by
temporal and spatial blocking techniques – it also relies on the
choice and development of adequate mathematical models that express
multilevel-parallelism and scalability up to a huge number of
cores. The models need to be adapted to a wide range of platforms and
programming environments. Data and problem decomposition on
heterogeneous platforms is a further objective. Memory transfer
reduction strategies are an important concept for facing the
bottlenecks of current platforms. Since hardware reality and
mathematical cognition give rise to different implementation concepts,
hardware-aware computing means also to find a balance between the
associated opponent guiding lines while keeping the best mathematical
quality (e.g. optimal work complexity, convergence order, error
reduction and control, accuracy, robustness, efficiency). All
solutions need to be designed in a reliable, robust and future-proof
context. The goal is not to design isolated solutions for particular
configurations but to develop methodologies and concepts that
preferably apply to a wide range of problem classes and architectures
or that can be easily extended or adapted. The \hiflow{} project has
implemented related concepts in the framework of the local
multi-platform LAtoolbox \cite{HLW2010,HSLW2010} within the Linear 
Algebra module.

\subsection{Object-oriented Programming Approach}
In order to be able to handle the complexity of
solving such a great variety of problems \hiflow{} is implemented in
C++. The object-oriented paradigm of C++ with its support for
polymorphism and inheritance allows all involved developers to
contribute with their individual specialized knowledge. End-user and
researcher can handle opaque and problem-oriented objects, computer
scientists and hardware specialists produce optimized low-level
implementations, while mathematicians provide new numerical solver
algorithms. Furthermore, the intensive use of the template features of
C++ gives the compiler a lot of opportunities for compile-time
optimizations. Our modular approach with utilization of the
object-oriented concepts of C++ is the best way to handle the
complexity of such a sophisticated software project with many
contributors. The approaches of data abstraction, encapsulation and
clear interfaces between various modules not only ease maintainability
of the code but also enable reusability of specific parts for the
extension of functionalities and features. An additional benefit of
this approach is the possibility to use parts of the software as
stand-alone libraries or modules for other projects like e.g.~the
LAtoolbox. In such a way, our project provides building blocks for the
development of new solvers and the extension to new problem domains.

\subsection{Modeling and Workflow}
Many physical problems are modeled by means of partial differential
equations.  Typical examples are the Poisson equation (e.g.~for
electric or gravity fields), stationary or time-dependent convection
diffusion equations (e.g.~transfer of particles or energy), and the
Navier-Stokes equation (e.g.~fluid flow around an obstacle, simulation
of a tropical storm, air flow in the lung). A classical approach for
solving associated problems relies on discretization of the equations
in the domain of interest by means of finite element methods
\cite{brezzi2008,braess2007,ciarlet2002} on spatial grids.  Searching
for approximations in finite dimensional function spaces by means of
weak formulations results in linear or nonlinear systems of equations
that are typically solved by Newton-type methods, time-stepping
schemes and iterative solvers. Coupling between degrees of freedom is
mainly expressed by nearest neighbor interaction with a high degree of
locality. Due to the required accuracy of the discrete approximations
large numbers of degrees of freedom are required resulting in huge and
sparse matrices with bad condition numbers in general.

For the finite element solution procedure the following steps have to
be performed for mapping a problem modeled by PDEs to \hiflow. First,
the physical problem has to be expressed in terms of PDEs in an
adequate way.  The next theoretical step is to derive a weak
formulation in a variational sense. In a preprocessing step the domain
of interest is discretized by means of a finite element triangulation
and converted to a format readable by the mesh module. Once the
spatial grid is read in it can be adjusted e.g.~by means of a
refinement procedure. Then, a problem-adapted finite element space
with appropriate ansatz functions has to be chosen. For the Navier-Stokes
equations Taylor-Hood elements are an appropriate choice. Once these two
informations are provided the degrees of freedom can be determined first locally
(this corresponds to the FEM part) on each cell, and then globally for the whole
mesh (this corresponds to the DoF part). In the case of distributed
memory platforms the DoF-partitioner is used to create a global degree of freedom (DoF) numbering
throughout the whole computational domain by only using its local information
and MPI communication across the nodes. Once all DoFs are identified the matrix
can be assembled by local integration over all elements within the
triangulation. Data structures for matrices and vectors are provided by the
Linear Algebra module. In case of nonlinear problems a Newton method is
combined with a linear solver for the problem solution. Finally, output in
either sequential or parallel format is provided for the visualization. 
This is an important aspect of the simulation cycle for an assessment and
exploration of simulation data. There exist several visualization tools which 
can be used for this postprocessing step. Within \hiflow{} there are
e.g.~back ends for HiVision \cite{HiVision} and Paraview \cite{Paraview}.

\subsection{Modules in \hiflow{}}
The implementation of a parallel and flexible finite element software package
aiming at general purpose deployment and portability across a wide range of
platforms comes along with various challenges in the respective
modules. 

% The mesh module takes care of the handling of the mesh. \hiflow{} is
% designed in first place for unstructured meshes. This results in the task to
% select an adequate data structure to store all necessary neighborhood
% relations. Distributed data structures are implemented for which reason the 
% connection of the local mesh on each processor in the global context must be
% provided without direct access to the data. 
% This fact results in the question of optimization of the communication between
% processes. Regarding the communication in matrix-vector operations implemented
% in the Linear Algebra module the concept of ghost cells is utilized. Each
% process gets additional information on the neighboring layer with respect to
% its local boundary cells. Another challenge arises from refinement and load
% balancing. The mesh offers the possibility to refine each cell locally with
% different types of refinements. In order to be able to track the history of
% refinement the parent relationship is stored.  
% The mesh itself can handle mixed meshes in two and three dimensions.
% Implemented cell types are triangles and quadrilaterals in 2D and 
% tetrahedrons and hexahedrons in 3D. Notably, there is no limitation to add more
% elements such as pyramids or prisms. The handling of non-conforming grids within 
% distributed data structures is considered e.g. for the ability to apply
% $h$-refinements locally to a cell \cite{Demkowicz,Schwab2004,Solin}.
The mesh module is responsible for the interaction with the
discretized computational domain. It is designed primarily for
unstructured meshes, and can handle several different cell types in
different dimensions. In order to support adaptive algorithms, the
Mesh module can work with both nonconforming meshes and meshes
containing cells of mixed types. Additionally, it provides
functionality to refine and coarsen a mesh, both globally and locally,
and to retrieve the history of these modifications. Furthermore, the
use of meshes distributed over several processors is supported, in
order to reduce the memory requirements for large-scale simulations
running on a high-performance cluster. For this purpose, the module
also contains functionality for dealing with partitioning and
communication of the mesh data. The link between the local mesh and
its neighbors is a layer of ghost cells that are
shared between each pair of processors.

The DoF/FEM module treats the problem of numbering and interpolating the degrees
of freedom. In the first step the local DoFs of each cell
for a chosen finite element space need to be determined. Then the local DoFs
of each cell have to be numbered globally. We are 
using a generic approach handling all cell types and finite element spaces.
In the case of distributed data structure the DoF module further handles the
neighborhood relations between cells which are distributed on different
processors. To this end, the module takes advantage of ghost cells created by
the mesh module. Another task is to distribute the DoFs in a balanced way such
that each processor is responsible for almost the same number of degrees of
freedom. Here, different complexities and work load contributions need to be
considered since each cell might use different ansatz functions.

The Linear Algebra module provides distributed and platform-optimized
data structures for vectors and matrices as well as implementations for corresponding
matrix-vector and vector-vector operations. Due to localized
interactions between finite element basis functions resulting matrices
are typically sparse. For that purpose, adequate sparse matrix formats
are provided. The structure of the Linear Algebra module is based on
two levels: the inter-node level communication layer utilizes MPI and
the intra-node communication and computation model is based on
platform-specific programming environments that are accessed by
generic and unified interfaces. In order to reduce the cost of
communication on the upper layer the Linear Algebra module takes
advantage of ghost cells provided by the Mesh module (and also used in
the DoF module). On the local intra-node layer the focus is set to
methods of hardware-aware computing aiming at best choice of
platform-specific implementations. The concept of the Linear Algebra
module allows to compute local matrix vector operations on hybrid
systems and accelerators such as multi-core-CPUs, GPUs, and
OpenCL-enabled devices. By providing unified interfaces with back ends
to different platforms and accelerators the module allows seamless
integration of various numerical libraries and devices. The user is
freed from any particular hardware knowledge -- the final decision on
platform and chosen implementation is taken at run time. Naturally,
scalability plays an important role in this module.

\subsection{Short Survey on Existing FEM-Software}
There exist reams of other non-commercial and commercial parallel
finite element software packages beside \hiflow{}. We do not want to
give an complete and exhaustive overview on state-of-the art finite
element software but we want to mention in alphabetical order at least a few of them.
Alberta \cite{Alberta} is an adaptive hierarchical finite element
toolbox. Only simplices are used for the triangulation. Refinement and
coarsening is restricted to bisection.  The main focus lies on
adaptivity which is achieved by different mesh modification algorithms
and a data structure which stores the mesh in a binary tree. COMSOL Multiphysics
\cite{COMSOL} is a commercial simulation software
environment. It includes the definition of the geometry, meshing,
specifying the physics, solving as well as the visualization.  For
good usability many common problem types are predefined as templates.
The C++ programming library deal.II \cite{BHK,BHK07} is an Open Source
project and uses adaptive finite elements for solving partial
differential equations. It supports one, two, and three space
dimensions, and is restricted to intervals, quadrilaterals and
hexahedrons. 
%Parallelization is possible only via MPI. 
DUNE \cite{DUNE} is a modular toolbox for
solving PDEs with grid based methods.  It is not restricted to Finite
Elements but also discretizes with Finite Volumes and Finite
Differences. The main principles are abstract interfaces, generic
programming techniques and reuse of existing finite element packages.
FEAST \cite{FEAST} is a software package designed to solve FE
problems. For better floating point performance and memory utilization
it can use different co-processors platforms like GPU and Cell
BE. It is based on a specific structure grid which lead to sparse
banded matrices.  Fluent \cite{Fluent} is a commercial Computation
Fluid Dynamics (CFD) software package provided from Ansys Inc. It is
one of the most mature software on the CFD market. NASTRAN is a program for general
Finite Element Analysis which was originally developed in the late
1960s for the NASA. The whole software package is quite large and
powerful but due to its age it is written in old legacy
languages. UG \cite{UG} stands for unstructured grids and is a software tool for
numerical solutions of partial differential equations. To achieve
numerical efficiency UG uses adaptivity on the grid level, multi-grid
methods and parallelism in form of distributed dynamic data. In most
cases based on the FE discretization, useful packages for solving the
obtained linear system are PETSc
\cite{petsc-user-ref,petsc-web-page,petsc-efficient} and Trilinos \cite{Trilinos}
which are highly
scalable linear algebra libraries providing data structures and routines
for large-scale scientific applications modeled by partial
differential equations. Their mechanisms are optimized for parallel
application codes, such as parallel matrix and vector assembly
routines and linear solving routines, and allows the user to have
detailed control over the solution process.
