\section{Conclusion}

The creation of a multi-purpose finite element software package that is
portable across a wide variety of platforms including emerging technologies
like hybrid CPU and GPU platforms is a challenging and multi-facetted task.
By our modular approach that utilizes the concepts of object-orientation, data
abstraction, dynamic polymorphism and inheritance we have created a highly
capable piece of software that provides a powerful means for gaining scientific
cognition. By utilizing several levels of parallelism by means of a two-level
communication and computation model and following the concepts of hardware-aware
computing \hiflow{} is a flexible numerical tool for solving bleeding-edge
scientific problems on the basis of finite element methods optimized for high
performance computers. Furthermore, the modules Mesh, DoF/FEM and Linear
Algebra complemented by auxiliary methods provide a broad suite of
building blocks for development of modern numerical solvers and application
scenarios. The user is freed from any detailed knowledge of the hardware -- he
only has to familiarize with the provided interfaces and needs to customize the
available modules in order to adapt \hiflow{} to his domain-specific problem
settings. As an open source project \hiflow{} further supports extensibility
of modules and methods. Within large scale projects like e.g.~the United Airways
project \hiflow{} has already proven its potential. In the next steps the efficiency
of the methods considered in the different modules especially related to the scalability
will be evaluated. Based on these results \hiflow{} will improve, following further 
the path of object oriented techniques in software design for scientific computing.
